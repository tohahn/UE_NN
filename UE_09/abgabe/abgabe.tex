\documentclass{article}
\usepackage{listings}
\usepackage[utf8]{inputenc}
\usepackage{amsmath}

\lstset{
	basicstyle=\footnotesize,
	numbers=left,
	tabsize=3,
	title=\lstname,
	breaklines=true
}

\addtolength{\oddsidemargin}{-.875in}
\addtolength{\evensidemargin}{-.875in}
\addtolength{\textwidth}{1.75in}

\addtolength{\topmargin}{-.875in}
\addtolength{\textheight}{1.75in}

\title{Neuronale Netze - Übung 9}
\author{Tobias Hahn\\ 3073375}	
	
\begin{document}
\maketitle
\newpage
\section{LSTM}
\subsection{Aufgabenstellung}
Die Aufgabenstellung war es, ein LSTM zu konstruieren welches 10 Eingaben nach einer 1 diese wieder ausgibt, wenn dazwischen nur 0en waren. So würde z.B. folgende Eingabe die Ausgabe 1 hervorrufen: 10000000000. Immer, wenn 10 0en auf eine 1 folgen wird also eine 1 ausgegeben, ansonsten 0.
\paragraph{}
Um diesen Effekt mit einer LSTM zu erreichen verwenden wir eigene Aktivierungsfunktionen und folgende Features: c, i und f. Die Aktivierungsfunktionen für die Gates werden jeweils mit angegeben.

\begin{align*}
	f^t &= 1 - x^t \\
	i^t &= 1 - x^t \\
	y^t &= \begin{cases}
		1 \text{ für } c^t = 10 \\
		0 \text{ sonst}
	\end{cases} \\
	c^t &= i^t + f^t * c^{t-1}
\end{align*}
\paragraph{}
Dieser Automat addiert ständig 1 zum Status hinzu, solange der Input 0 ist, wenn der Input einmal 1 ist wird der Status vergessen und der Counter somit neu gestartet. Um sicherzugehen dass keine 1 ausgegeben wird obwohl noch keine 1 kam muss der Counter initial auf -unendlich gesetzt werden.

\subsection{Rekurränte McCulloch Pitts Zellen}
Ja, diese Aufgabe kann auch mit diesen Zellen erledigt werden. Dabei können wir einen FSM erstellen der 10 Zustände und einen Startzustand hat. Vom Startzustand kommt man mit 1 auf den ersten Zustand, mit 0 wieder auf den Startzustand. Von da an kommt man von Zustand 1 bis 9 jeweils mit 0 zum weitern Zustand, mit 1 wieder zurück zu Zustand 1. Bei Zustand 10 kommt man mit 1 zurück zu Zustand 1 und mit 0 zurück zum Startzustand. Es wird in jedem Zustand 0 ausgegeben außer im Zustand 10, wo 1 ausgegeben wird.
\paragraph{}
Diesen FSM können wir mit der in der Vorlesung vorgestellten Methode in ein Netz aus McCulloch Pits Zellen verwandeln.

\section{LSTM Train}
Das LSTM wurde mit Keras trainiert, welches intern Tensorflow benutzt. Dafür wurde Random-Data erstellt und dann trainiert.
\subsection{Code}
\subsubsection{Data Generation}
\lstinputlisting[language=Python]{"../data/gen_data.py"}
\subsubsection{Training}
\lstinputlisting[language=Python]{../code/lstm.py}
\subsection{Results}
\subsubsection{Console Output}
\begin{lstlisting}
Using TensorFlow backend.
Epoch 1/1
7940s - loss: 5.0080e-04
Saved model to disk
\end{lstlisting}
\subsubsection{Saved Model}
\lstinputlisting{../code/model.json}
\subsubsection{Prediciton and error evalutation}
\lstinputlisting{../code/test.py}
\lstinputlisting{../code/error.py}
\paragraph{Console output}
Mean squared error: 0.00045656099664777614
\end{document}
