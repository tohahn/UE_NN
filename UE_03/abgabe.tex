\documentclass{article}
\usepackage{amsmath}
\usepackage[utf8]{inputenc}
\begin{document}
\section*{Übung 3 - Neuronale Netze}
\subsection*{Tobias Hahn - 3073375}
\subsection*{XOR Funktion}
\paragraph{}
Zuerst zeichnen wir die XOR Funktion mit Belegungen auf, um zu ermitteln für welche Belegungen unsere Gewichte welches Ergebnis liefern müssen.
\paragraph{}
\begin{tabular}{|l|l|l|}
	\hline
	\( x_1 \) & \( x_2 \) & \( x_1 \oplus x_2 \) \\\hline
	0 & 0 & 0 \\\hline
	0 & 1 & 1 \\\hline
	1 & 0 & 1 \\\hline
	1 & 1 & 0 \\\hline
\end{tabular}
\paragraph{}
Nehmen wir nun zwei beliebige aber fixe Gewichte \( w_1 \) und \( w_2 \) sowie einen beliebigen aber fixen Schwellwert \( \theta \) an, so erhalten wir folgende Gleichungen:
\begin{alignat*}{1}
	w_1 * 0 + w_2 * 0 &<  \theta \\
	w_1 * 0 + w_2 * 1 &\ge  \theta \\
	w_1 * 1 + w_2 * 0 &\ge \theta \\
	w_1 * 1 + w_2 * 1 &< \theta
\end{alignat*}
\paragraph{}
Formen wir nun die letzen drei Gleichungen in Aussagen über die Gewichte um, so kommen wir zu unserem Widerspruch:
\begin{alignat*}{1}
	w_2 &\ge \theta \\
	w_1 &\ge \theta \\
	w_1 + w_2 &< \theta
\end{alignat*}
\paragraph{}
Da es nicht sein kann dass zwar \(w_2\) und \(w_1\) jeweils größer sind als \(\theta\), sie zusammengerechnet jedoch kleiner sind, gibt es keine Gewichte \(w_1\), \(w_2\) und \(\theta\) die diese Gleichungen erfüllen können, damit ist gezeigt dass die XOR Funktion mit einem einfachen Perzeptron nicht berechnet werden kann.
\end{document}
